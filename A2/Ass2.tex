\documentclass[12pt]{article}

\usepackage{fullpage}
\usepackage{fancybox}
\usepackage{amssymb}
\usepackage{charter}
\usepackage{verbatim}
\usepackage{graphicx}
\usepackage{amsmath}
\setlength{\textwidth}{7in}
\setlength{\evensidemargin}{-0.24in}
\setlength{\oddsidemargin}{-0.24in}
\setlength{\textheight}{9.45in}
\setlength{\topmargin}{-0.45in}
\setlength{\parindent}{0.3in}
\headheight12pt
\headsep16pt
\pagestyle{myheadings}
\newcounter{ques}
\newenvironment{question}{\stepcounter{ques}{\noindent\bf Question \arabic{ques}:}}{\vspace{5mm}}

\begin{document}

\begin{center} \large\bf
COMP 3804/MATH 3804\\
Design and Analysis of Algorithms I  -
Fall  2021\\
Assignment 2\\
Félix Cardinal Tremblay - 101141593
\end{center}

Hand in your assignments on, or before
Oct $23^{rd}$ 23:59. No late assignment will be accepted. Your assignment should be submitted online on Brightspace as a single .pdf file.  The filename should contain your name and student number. No late assignments will be accepted.You can type your assignment or you can upload a scanned copy of it.  Please, use a good image capturing device. Make sure that your upload is clearly readable. If it is difficult to read, it will not be graded. Whenever you are designing an algorithm you must address the three questions we are
typically posing (correctness, complexity and improvement potential).
The faster your
algorithm, the better your mark.     \\

\vspace{1em}

\begin{question}[10 points]\\
Consider you are given a heap, called $n-$heap, on $n=28$ elements and a heap, called $k-$heap, on k=9  elements. Show the
 descriptors for the pennant forests correspoding to the two heaps. Show the descriptor for the pennant forest for the heap, called $n+k-$heap,
resulting from applying the operation merge two heaps discussed in class (see also the paper on the course web-page). Illustrate, step by step,  how the merge operations works, if we are only! concerned about
the heap shape, i.e., NOT the heap order.\\\\

We will begin by analyzing the pennants of the 28-heap, in order to find the descriptor.\\
First, we must find the height of the tree, as well as which side will have a "complete" pennant. By calculating $log(28)=4.81$, we can see that since $4.81>4.5$, the last node must be on the right side of the tree, and so the left side of the tree will be a complete pennant. We then want to find the height of this tree, to calculate how many nodes it will have. Since $\lfloor log(28)\rfloor=4$, we know that the subtree must have a height of 3 (because the $0^{th}$ layer is the single node at the top of the pennant). Then, since our subtree has a height of 3, we know from class that the full pennant must have $2^{n+1}$ nodes where $n$ is the height of the tree, and thus we can currently show our descriptor as : $(1,\_,\_,\_,\_)$.\\
Next, we will look at the left subtree. This will be a tree containing $28-16=12$ nodes, and so we can do $log(12)=3.58$, and now we know that the left side of the tree will once again be a full pennant. The height of the subtree (not including the top node of the pennant) will be 2, and so the total amount of nodes in that subtree will be $2^{3}$. Now, our descriptor looks likes the following: $(1,1,\_,\_,\_)$\\
The next subtree has a size of $12-8=4$. Since 4 is a relatively small number, it is very easy to visualize the remainder of the question. A heap with four nodes will have one node in the top layer, two nodes in the layer below that, and one node in the leftmost position in the last layer. Therefore, we can see that there will be a pennant of size two consisting of the top node and the rightmost node, and two single-node pennants consisting of the last two nodes. Therefore, our final descriptor for the $28$-heap will be:
$$(1,1,0,1,2)$$

Next, we need to analyze the 9-heap. We can begin by seeing that $log(9)=3.2$. Therefore, the subtree on the right of the heap (not counting the node at the top of the pennant) will have a height of one (since there are no nodes on the last level of the tree on the right side), and so including the pennant, there will be $2^{n+1}=2^2$ nodes in this subtree, where $n$ is the height of the tree. We know have a descriptor in the following form: $(1,\_,\_)$.\\
Now, we can look at the left side of the heap, which has $9-4=5$ nodes. Again, this tree is quite trivial to visualize, and so we know that the top-most and right-most nodes will create a pennant of size two, and the final three nodes will create a pennant of size two and a pennant of size one. Therefore, our final descriptor will be
$$(1,2,1)$$\\\\

Now, we must show the descriptor for the $37-$heap.
First, we want to find the first side of the heap that we can use to create the pennant. We know that $log(37)=5.2$, so we'll be looking at the right side of the heap to create the first pennant. Not including the first node of the pennant, it will have a height of 3 (due to the fact that on the right side, there are no nodes on the bottom-most layer of the tree), and so including the pennant, it will contain $2^4$ nodes. Thus, our descriptor will now be $(1,\_,\_,\_,\_)$\\
Next, we will look a the subtree of size $37-16=21$. Since $log(21)=4.39$, we know that the right side will once again have a pennant. Thus, we can see that the bottom layer of the right side of the tree will once again not have any leaves, and so the subtree (not including the pennant) will have a height of 2. Therefore, the entire pennant must have $2^3$ nodes. Thus, our descriptor is now $(1,1,\_,\_,\_)$\\
The next subtree will have $21-8=13$ nodes, and so we can show that $log(13)=3.7$, and so the left side will have a full pennant. The subtree of the pennant will have a height of 2, and so it will contain $2^3$ nodes. Thus, our descriptor is now $(1,2,\_,\_,\_)$.\\
Next, we will be examining a subtree with $13-8=5$ nodes. We've already examined one of this size in the previous parts of the question, so we know that it will have two pennents of size two, and one of size one. Therefore, our final descriptor for the pennant forest is:
$$(1,2,0,2,1)$$
Now, we must build the $37-$heap tree using the pennants we have found from the $28-$heap and the $9-$heap, and to do so we will process the $37-$heap's descriptor from right to left.\\
First, we see that there is a single pennant of size $2^0$. We know that there is one pennant of size 1 in the in the $9-$heap and two nodes of size 1 in the $28-$heap, and so we could take them from either heaps arbitrarily. We will take it from the $9-$heap, however the same process would work if we were to take it from the other heap.\\
Next, we need two pennants of size 2. We can notice that once again, we have two in the $9-$heap, and so we will use those ones, although we could again use either of them arbitrarily. Now, we need pennants of size 8 and 16, however we have leftover pennants, and so we can merge them to create those. We merge take the two pennants of size 1 and the pennant of size 2 from the $28-$heap to create a pennant of size 4. Then, we will merge this new pennant with the pennant of size 4 in the $9-$heap, to create a pennant of size 8. We now notice that we have a pennant of size 8 from the $28-$heap, a well as another pennant of size 8 that we have just created by merging smaller pennants. Since we need two pennants of size 8 in the $37-$heap, we can use these two. Lastly, we need a pennant of size 16, and we have a single pennant of size 16 in the $28-$heap. By using this final pennant, we have used all of the pennants, and successfully created the $37-$heap by merging a $9-$heap and a $28-$heap in a very efficient way.
\end{question}

\begin{question}[10 points]\\
The linear-time algorithm for selection discussed in class groups the n input elements into $n/5$ groups of $5$ elements each. Does a grouping into
$n/9$ groups of $9$ elements each, or $n/3$ groups of $3$ elements each, work? Argue exactly why, or why not, by reworking for these two instances the analysis carried
out in class.\\\\

Yes, it would still work. This is because we are splitting up the input to conduct multiple operations in constant time.\\\\

We will begin by explaining the steps required in order to execute the algorithm on a set where there are $\lceil\frac{n}{9}\rceil$ partitions of size 9.\\
The first step, of course, would be to partition the set into $\lceil\frac{n}{9}\rceil$ smaller sets. This would simply be $O(1)$ time, due to the fact that it is a very rudimentary operation.\\
Next, we would then sort each of the partitions. Since there are $\lceil\frac{n}{9}\rceil$ partitions and each sorting is done in $O(1)$ time, the total time taken for that operation would be $O(1)\cdot\frac{n}{9}$\\
Then, we want to find the median of the medians (called $x$), and since we have $\frac{n}{9}$ medians, this would take $\frac{n}{9}$ steps.\\
Then, we want to partition the input array around $x$.
We will let  $j=1+$ The number of elements less than $x$, and we will call that set $S_L$
We will also let put all of the remaining elements (which are greater than or equal to $x$) in a set $S_R$.
basically, we'll be setting it in a way where all of the partitions to the left are lesser than the median of medians, and the partitions to the right are greater than the median of medians.\\
So, now we know that all of the elements that are lesser than the medians to the left are going to be lesser than $x$.\\
Additionally, we now know that all of the elements greater than the partitions that are to the right are going to be greater than the median of medians.
So, for all the groups to the right, 5 of the elements will definitively be greater than $x$, and in the group that contains $x$, 4 elements will be greater than it.\\
The amount of groups that will be on the right side will be $\lceil\frac{n}{18}$.
We will want to "throw out" a constant number of elements each time we recurse on the sets. Since we know that there will always be 5 elements that will be greater than $x$ in $S_R$, except for the partition that contains $x$, as well as the last partition, if the input list is not a multiple of 9. So, a conservative estimate would show us that we will be able to "throw out" $5(\lceil\frac{n}{18}\rceil-2)$ elements for each iteration of the algorithm. Therefore, we have:
$$5(\left\lceil\frac{n}{18}\right\rceil-2)>\frac{5n}{10}-10=\frac{n}{2}-10$$
And this means that when we call the algorithm again, instead of executing it on $n$ elements, we will be executing it on
$$n-\left(\frac{n}{2}-10\right)=\frac{n}{2}+10$$
Then, we can show the following:
\begin{align*}
  T(n) &\geq c\cdot\left\lceil\frac{n}{9}\right\rceil + c\left(\frac{n}{2}+10\right)+O(n)\\
       &< \frac{cn}{9} + c + \frac{cn}{2} + 10c + O(n)\\
       &< \frac{11cn}{18} + 11c + O(n)\\
       &< c\cdot n \text{    For some $c$.}\\
       &  \text{Which, by definition is } O(n).
\end{align*}
Next, we want to show the same thing, except for partitions of size $3$ rather than of size $9$. Since the explanation will be extremely similar, we will avoid repetition for the sake of avoiding redundancy. We will begin by noting the following:\\
\begin{enumerate}
  \item We will partition the set into $\frac{n}{3}$ subsets of size $3$, and find their medians.
  \item For subsets that we "throw out" when recursing, each partition to the right of the partition containing $x$ will have two elements that are greater than $x$, exept for the last one if the set is not a multiple of 3.
\end{enumerate}
This means that we can begin the process with the same equations we had before, but with slightly different numbers. We can show that for each iteration, we will be throwing out $2(\lceil\frac{n}{6}\rceil-2)$ elements, which we can write as:
$$2(\left\lceil\frac{n}{6}\right\rceil-2)>\frac{2n}{6}-4=\frac{n}{3}-4$$
And so, removing that many elements, we know that we will be recursing on a set of size
$$n-(\frac{n}{3}-4)=\frac{2n}{3}+4$$
Next, we can continue like so:
\begin{align*}
  T(n) &\geq c\cdot\left\lceil\frac{n}{6}\right\rceil+c\left(\frac{2n}{3}+4\right)+O(n)\\
       &< \frac{cn}{6}+c+\frac{2cn}{3}+4c+O(n)\\
       &< \frac{5cn}{6}+5c+O(n)\\
       &< c\cdot n \text{    For some $c$.}\\
       &  \text{Which, by definition is } O(n).
\end{align*}
So, we have successfully shown that the Selection algorithm discussed in class also works if the elements were grouped into partitions of size 3 or 9 instead of size 5.
\end{question}

\begin{question}[10 points]\\
Suppose you want to support the operation DeleteAnyElement(pointer into the heap to the element to be deleted) in a heap. You must do this efficiently.
Describe your algorithm in sufficent and establish its correctness.\\\\
One thing to keep in mind is that according the the TA Tyler Tuttle, we know the size of the heap. This means we can find the last element very easily.\\
We will assume, for the sake of simplicity, that we are working with a min heap.

\begin{verbatim}
DeleteAnyElement(Element e){
  replace the element with the last element in the heap
                      (removing it from the last position)
  if the new element is less than its parent {
    bubbleUp(e);
  } else if the element is greater than either of its child values{
    bubbleDown(e);
  }
}

bubbleUp(Element e){
  if the element does not have a parent, return.

  if the element's value is lesser than its parent's value {
    switch the element and its parent
    bubbleUp(e);
  }else {
    return;
  }
}

bubbleDown(element e){
  if the element does not have any children, return.

  let x = the lesser of the two children of e, or the only child if there is only one
  if e's value is greater than x{
    swap e and x.
    bubbleDown(e);
  }
  else{
    return;
  }
}
\end{verbatim}
We need to prove that our code is correct. We will first prove the functions \textbf{bubbleUp()} and \textbf{bubbleDown()} to be correct. First, we will look at \textbf{bubbleUp():}\\
\textbf{Base case:}\\
Let us examine a heap of size 1:\\
The element passed into the function will be the root node of the heap, and not have any parents. Therefore, our base case is valid.\\\\
\textbf{Inductive step:}\\
Now, we will examine a heap of size $n>1$, where $e$ is not a root node.\\
First, the element will be compared against the value of its parent. If the parent has a lesser value, the function will simply return, which is correct.\\
If the parent has a greater value, then the elements will will be switched (which is valid since if $e$ is greater than the current max, it must be the max of that subtree), and we will then recursively call \textbf{bubbleUp()}. Since there is one less element to be recursed upon, by the inductive hypothesis we can assume that bubbleUp() will be correct, since the size of the heap that is analyzed will be at least $n-1$. This is because we are only analyzing from $e$ up to the root node.\\
Therefore, bubbleUp() must be correct.\\\\
Now, we will prove the correctness of \textbf{bubbleDown():}\\
\textbf{Base case:}\\
Let us examine a heap of size 1:\\
The element will not have any children, and so it will return, which is correct. Thus, our base case is valid.
\textbf{Inductive step:}\\
Now, we will examine a heap of size $n>1$, where $e$ is not a leaf node.\\
First, we will find which of the children is the smallest (or the only leaf, if there is only one). This is because if we are to switch spots with the smallest of the two children, we know that the new "min" value will be correct. Then, if the current node is smaller than one of its children, we swap it with the smallest of the two. Then, we call \textbf{bubbleDown()} again, which by the inductive hypothesis, must correctly bubble the element down to its correct position in the heap.\\\\
Now, we can finally prove the correctness of the full algorithm:\\
First, we replace $e$ with the last element in the tree. This means the element we wanted to delete is now gone, and since we removed the last element, there is no issue with the structure of the tree. This new element may be in the wrong position though. This is fixed by the following \textbf{if-statement}, as if it belongs in a higher position it will be bubbled up, and if it belongs in a lower position, it will be bubbled down. Due to the fact that we have already proven the correctness of these functions, we know that our code must be correct.\\\\
\textbf{Runtime analysis:}\\
The first line simply replaces a value, which is $O(1)$.\\
Next, either of the bubbleUp or bubbleDown fucntions will be called. These functions work by doing a $O(1)$ operation, and recursively either going up or down a level in the heap. This means that since the depth of the heap is $\lceil log(n)\rceil$, the maximum amount of work done will be $O(log(n))$. Therefore, the total amount of work for \textvf{DeleteAnyElement()} must be at most $O(log(n))
\end{question}



%YOU MAY NEED TO ADD A BASE CASE FOR N>1, NOT 100% SURE THOUGH








\begin{question}[10 points]\\
Consider a permutation of the numbers 1, ..., n as input to the following algorithm:\\

\noindent Initialize an empty stack;

\noindent {\bf For} each input value x:

\indent	{\bf While} the stack is nonempty and x is larger than the top item on the stack {\bf do}

\indent \indent  pop the stack to the output

\indent	Push x onto the stack

\noindent {\bf While} the stack is nonempty {\bf  do} pop it to the output\\


What  is the sequence of POP/PUSH operations executed on input permutations:

a) 3214   b) 4123  ?


Find all permutations  of 4 input elements that the algorithm does not sort correctly?
What happens in this case? Characterize the permutations (i.e., see a pattern) in your own words?

\textbf{a)}
The sequence would be as follows:
\begin{itemize}
  \item First, the 3 will be pushed onto the empty stack.
  \item Then, the 2 will be pushed onto the stack since $2<3$.
  \item Then, the 1 will be pushed onto the stack since $1<2$
  \item Then,  since $4> 3$, 2 and 1, we will pop 1, then 2, then 3 onto the stack.
  \item Then, 4 will be pushed into the newly empty stack, and popped unto the output, which will properly sort the permutation.
\end{itemize}

\textbf{b)}
The sequence would be as follows:
\begin{itemize}
  \item First, the 4 will be pushed onto the empty stack.
  \item Then, since $1<4$, 1 will be popped to the output.
  \item Then, since $2<4$, 2 will be popped to the output.
  \item Then, since we have gone through the inputted permutation, 4 will be popped onto the output, which will properly sort the input.
\end{itemize}

\textbf{Permutations that are not sorted correctly:}\\
\begin{center}$
  \begin{matrix}
    &1342 &2341 &2313 &2413 &2431\\
    &3142 &3241 &3412 &3421 &4321
  \end{matrix}$
\end{center}


\textbf{Characterization of the permutations:}\\
Let two consecutive numbbers of a permutation be a pair.\\



There must be at least one ascending pair where the first number of that pair cannot be equal to $n+1$ (where $n=$ the lesser value of the most recent ascending pair. If there is no previous ascending pair, $n=0$). Additionally, (if the first ascending pair does not fulfill the previous requirements) if an ascending pair's first number $=n+2$, then the number to the left of the the first pair's lesser number cannot be $n+1$ or $n-1$\\
Finally, the first number of that pair cannot be the $xth$ value (where $x=$ the position of the larger value of the previous ascending pair) if it is the largest of all previous numbers.\\
We will now show an example:\\\\
if the permutation were 654123, this would be sorted correctly, as the first ascending pair would be "12". $1=n+1=0+1=1$, and so this pair would not cause it to be sorted incorrectly.\\
The next ascending pair would be "23", and since $2=n+1=1+1=2$, this would once again not cause it to be sorted incorrectly. There are no other ascending pairs in the permutation, and so this would not cause any issues in the algorithm.\\\\
If instead we were to take the permutation "652134", the first pair would be "13", where we have the same situation as before. However, since the second pair's lesser number $=n+2$ and the number to the left of the first pair's lesser number $=n+1$, this will be sorted without any issues.\\\\
Finally, if we were to have the permutation "7431563", we can see that we have \textbf{1:} we have an ascending pair where the first number is not $n+1$ (as with the pair "56", $n=1$). Then, we have \textbf{2:} $5\neq n+2$, and finally \textbf{3:} since $7>5$, $5$ is not the largest number when compared to all previous numbers.\\\\
\textbf{Note:} I answered this question while under the impression that we had to find a pattern in terms of inputs from $1-n$ rather than just $1-4$. This does over-complicate the explanation a bit, however it is still valid.

\end{question}

 \newpage

 \begin{question}[10 points]\\

 \begin{enumerate}
 	\item
 	Suppose we want to find the minimum spanning tree of the graph shown below on Figure~\ref{fig:MSTgraph}. Run Prim's algorithm; whenever there is a choice of nodes, always use alphabetic ordering (e.g., start from node A). Draw a table showing the intermediate values of the cost array.
 	\item Run Kruskal's algorithm on the same graph. Show how the disjoint-sets data structure looks at every intermediate stage (including the structure of the directed trees), assuming path compression is used.
 \end{enumerate}


\end{question}


	\begin{figure}
		\centerline{\resizebox{!}{0.25\textwidth}{\includegraphics{MSTExample.pdf}}}
		\caption{Input graph for Question 5}
		\label{fig:MSTgraph}
	\end{figure}

\end{document}
