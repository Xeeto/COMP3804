\documentclass[12pt]{article}

\usepackage{fullpage}
\usepackage{fancybox}
\usepackage{amssymb}
\usepackage{charter}
\usepackage{verbatim}
\usepackage{graphicx}
\setlength{\textwidth}{7in}
\setlength{\evensidemargin}{-0.24in}
\setlength{\oddsidemargin}{-0.24in}
\setlength{\textheight}{9.45in}
\setlength{\topmargin}{-0.45in}
\setlength{\parindent}{0.3in}
\headheight12pt
\headsep16pt
\pagestyle{myheadings}
\newcounter{ques}
\newenvironment{question}{\stepcounter{ques}{\noindent\bf Question \arabic{ques}:}}{\vspace{5mm}}

\begin{document}

\begin{center} \large\bf
COMP 3804/MATH 3804\\
Design and Analysis of Algorithms I  -
Fall  2021\\
Assignment 3
\end{center}

Hand in your assignments on, or before
Nov $14^{th}$ 23:59. No late assignment will be accepted. Your assignment should be submitted online on Brightspace as a single .pdf file.  The filename should contain your name and student number. No late assignments will be accepted.You can type your assignment or you can upload a scanned copy of it.  Please, use a good image capturing device. Make sure that your upload is clearly readable. If it is difficult to read, it will not be graded. Whenever you are designing an algorithm you must address the three questions we are
typically posing (correctness, complexity and improvement potential).
The faster your
algorithm, the better your mark.     \\

\vspace{1em}

\begin{question}[5 points]\\
Dijkstra's algorithm is not applicable  for negatively weighted edges. So, why don't we just "scale everything up" by adding to each edge-weight: one plus  the absolute value of the globally smallest (negative) weight. This way, all weights are positive. Does the algorithm, when applied to the modified graph,  correctly compute the shortest path in the orginal graph? If so, give a proof;   if not, give a counter-example.\\\\

Assume we are currently executing the algorithm, and the vertex that was just removed from $V$ and added to $S$ is $v$.\\\\
Since we have just removed $v$ from $S$, we must assume that $d[v]=\delta(u,v)$, where $u$ is our "starting" node. However, assume there is a node $z$ that is connected to $v$, and  $d[v] > d[z] + w(z,v)$. Additionally, assume that if $(z,u)$ did not exist or its weight was positive, $\delta(u,z)>\delta(u,v)$.\\\\
By these assumptions, we know that $v$ will be put into $s$ before $z$, and so we will not be able to "update" the value of $d[v]$ to include the path that considers going through $d[z]$ and using the negative edge $(u,v)$ in order to attain $d[v]$. This shows that the graph cannot handle negative edges. \textbf{However}, if we were to instead add one plus the absolute value of the globally smallest weight to all edges, we would know from our assumption that $\delta(u,z)>\delta(u,v)$ and so $w(z,v)$ would be irrelevant, as it cannot be negative and would not change our equality. Because $d[v]=\delta(u,v)$ when $v\in S$, we know that there could not be any other path which would be able to reach $v$ from $u$ with a lesser weight.\\\\
Therefore, by adding the absolute value of the global minimum plus one to the weight of every edge, Dijkstra's algorthm would run properly.

% if there is a negative that connects to $v$, then that means that we may be able to go through a node $z$
%
%  (one that is taken out of the set $V$ later), and then arrive at the same vertex while it being lesser than the value of $d[v]$. However, the algorithm does not allow for a change in $d[v]$ once $v\in S$, and so it will be invalid.\\\\
% By adding the absolute value of the globally smallest edge weight +1, we are able to ensure that all weights are positive. This means that in our prior situation, we know that when $v$ is added to $S$, it must be the path with the smallest weight to $S$. Therefore, if we were to choose any other path, it would automatically be greater that $d[v]$, and so we will not care about its value, even if it is very small. because it is no longer negative, it would be impossible to have a case where a smaller path is found, after $v\in S$.




\end{question}

\begin{question}[20 points]\\
 Let us assume that we have a directed graph G =(V,E) with $|V|$  = $n$ nodes of which there are $k$ starting nodes from which we would like to know their shortest path distances to a common destination node (many-to-one problem).

 \begin{itemize}
 	\item We could solve the problem by applying Dijkstra's algorithm $k$ times, ones for each of the $k$ starting nodes. What is the time  complexity (stated in terms of $n$ and $k$)?
 	\item Alternately, we could  start at the destination node and  somehow go backwards. Describe how this would work, i.e., how would we modify Dijkstra's algorithm and/or its input to achieve this?  Then, state the time complexity of this solution to our original problem. (Do not forget to argue why the algorithm, as modified, is correct!)
 \end{itemize}
\end{question}

\begin{question}[15 points]\\
	Suppose we want to compute the shortest path tree from node, A,  using Bellman-Ford for the graph shown below in Figure~\ref{fig:Bellman-Ford}. Show how Bellman-Ford would work on this graphs as input.

\newpage
\begin{figure}
	\centerline{\resizebox{!}{0.45\textwidth}{\includegraphics{BellmanFord.pdf}}}
	\caption{Input graphs for Question 3}
	\label{fig:Bellman-Ford}
\end{figure}

 %~\ref{fig:Bellman-FordTable}
\begin{figure}
	\centerline{\resizebox{!}{0.3\textwidth}{\includegraphics{Bellman-FordTable.pdf}}}
	\caption{Solution format for the graphs Figures 1  A,B}
	\label{fig:Bellman-FordTable}
\end{figure}
\textbf{Answer for the graph of Figure A:}\\
The edges we will analyze will be in the following order:
$$(B, D), (B, E), (E, C), (C, B), (D, C), (A, D), (A, B)$$
\begin{center}
  \begin{tabular}{||c | c c c c c||}
   \hline
    step \# & A & B & C & D & E \\ [0.5ex]
   \hline\hline
   0 & 0 & $\infty$ & $\infty$ & $\infty$ & $\infty$\\
   \hline
   1 & 0 & $\infty$ & $\infty$ & 3/A & $\infty$\\
   1 & 0 & 40/A & $\infty$ & 3/A & $\infty$\\
   \hline
   2 & 0 & 40/A & $\infty$ & 3/A & 35/B\\
   2 & 0 & 40/A & 37/E & 3/A & 35/B\\
   2 & 0 & 40/A & 2/D & 3/A & 35/B\\
   \hline
   3 & 0 & 12/C & 2/D & 3/A & 35/B\\
   \hline
   4 & 0 & 12/C & 2/D & 3/A & 7/B\\
   \hline
  \end{tabular}
\end{center}
And thus, since we have $|V|=5$ and we have executed the "loop" of the algorithm $|V|-1$ times, we have completed the first part of the algorithm.\\\\
Now, we can do the second part of the algorithm, which checks for negative cycles.\\
For each edge $(u,v)$ in $E$, we check if $d[v] > d[u] + w(u,v)$. So, we can show the following:
\begin{center}
  \begin{tabular}{||c|c|c|c||}
   \hline
    $(u,v)$ & $d[v]$ & $d[u] + (u,v)$ & cycle?\\ [0.5ex]
   \hline\hline
   $(B,D)$ & 3 & $12+20=31$ & no\\
   \hline
   $(B,E)$ & 7 & $12+(-4)=7$ & no\\
   \hline
   $(E,C)$ & 2 & $7+2=9$ & no\\
   \hline
   $(C,B)$ & 12 & $2+10=12$ & no\\
   \hline
   $(D,C)$ & 2 & $3+(-1)=2$ & no\\
   \hline
   $(A,D)$ & 3 & $0+3=3$ & no\\
   \hline
   $(A,B)$ & 12 & $0+40=40$ & no\\
   \hline
  \end{tabular}
\end{center}
Therefore, we know that there cannot be any negative cycles in the graph, and so the algorithm will return true.\\\\
We will now show our answer for Figure B:\\
The edges we will analyze will be in the following order:
$$(B, D), (B, E), (E, C), (C, B), (D, C), (D, A), (A, B)$$
\begin{center}
  \begin{tabular}{||c | c c c c c||}
   \hline
    step \# & A & B & C & D & E \\ [0.5ex]
   \hline\hline
   0 & 0 & $\infty$ & $\infty$ & $\infty$ & $\infty$\\
   \hline
   1 & 0 & -40/A & $\infty$ & $\infty$ & $\infty$\\
   \hline
   2 & 0 & -40/A & $\infty$ & -20/B & $\infty$\\
   2 & 0 & -40/A & $\infty$ & -20/B & -45/B\\
   2 & 0 & -40/A & -43/E & -20/B & -45/B\\
   2 & -17/D & -40/A & -43/E & -20/B & -45/B\\
   2 & -17/D & -57/A & -43/E & -20/B & -45/B\\
   \hline
   3 & -17/D & -57/A & -43/E & -37/B & -45/B\\
   3 & -17/D & -57/A & -43/E & -37/B & -52/B\\
   3 & -17/D & -57/A & -50/E & -37/B & -52/B\\
   3 & -34/D & -57/A & -50/E & -37/B & -52/B\\
   3 & -34/D & -74/A & -50/E & -37/B & -52/B\\
   \hline
   4 & -34/D & -74/A & -50/E & -54/B & -52/B\\
   4 & -34/D & -74/A & -50/E & -54/B & -79/B\\
   4 & -34/D & -74/A & -77/E & -54/B & -79/B\\
   4 & -51/D & -74/A & -77/E & -54/B & -79/B\\
   4 & -51/D & -91/A & -77/E & -54/B & -79/B\\
   \hline
  \end{tabular}
\end{center}
And thus, we have executed the loop of the algorithm  $|V-1|$ times. We must now execute the second loop of the algorithm, which checks for negative cycles.
We will check the first edge: $(B,D)$.
We can see that after our final execution of the code, $d[D] > d[B] + w(B,D)$ because $-54 > -91 + 20$. Therefore, the algorithm will return false due to the fact that there is a negative cycle.

\end{question}

\begin{question}[15 points]\\
Let $n$ be the number of vertices of a given triangulation of a convex polygon.

\begin{itemize}
	\item How many triangles does the triangulation consist of? Prove your statement by induction on $n$.
	\item Consider now the simplest algorithm for computing weighted shortest paths, as described in class which places the same number of Steinerpoints on each edge of the triangulation.   Assume that we place, say 7 Steinerpoints,  on each edge of the triangulation. Precisely, how many edges does the algorithm add? Argue!

\end{itemize}


\end{question}

\newpage


\begin{question}[15 points]\\
Consider the graph given in Figure 3 above.
\begin{itemize}
	\item Run DFS on the graph and classify each edge as being either: Tree edge, Forward edge, Back edge, or Cross edge. Show and argue: the algorithm execution,   pre(v) and post(v) time intervals and the edge-classification. (An edge type may or may not appear in a particular graph.)
	\item Find a topological order of the nodes or argue that no such order can exist.
\end{itemize}



\begin{figure}
	\centerline{\resizebox{!}{0.7\textwidth}{\includegraphics{DFSinput.pdf}}}
	\caption{Input for DFS algorithm}
	\label{fig:DFSinput}
\end{figure}

\end{question}
\end{document}
